\documentclass{article}
\usepackage{times}

\title{An Overview of EGOS: \\ The Earth and Grass Operating System}
\author{Robbert van Renesse}
\date{}

\begin{document}
\maketitle

\section{Quick Overview}

The Earth and Grass Operating System (EGOS) is a ``user mode operating system''
in that it runs inside an ordinary (Linux or Mac OS X) process.
Instead of approx.~17M lines of code in an OS like Linux, EGOS proper has
approx.~3,000 lines of code, making it relatively easy to wrap your brain around the
whole thing.

EGOS consists of three layers (bottom to top):
\begin{enumerate}
\item \emph{Earth} emulates an x86 hardware platform, replacing its rather
complicated low-level hardware management facilities with a simple
software-loaded TLB, interrupt handling,
(emulated) kernel/user mode, and a few simple devices such as a terminal,
clock, disk, and so on.
\item \emph{Grass} is a simple but working operating system microkernel that manages
multiple processes (including paging), files, directories, and so on.
\item On top of all this run the applications, and EGOS comes with some
simple Unix-style apps like a shell.
\end{enumerate}

Much of Grass is structured as a set of ``kernel processes'' that
interact via message passing.  (A Grass user process is a special
kind of kernel process that manages a virtual address space---more on
this later.)
The kernel processes implement simple services such as a disk service,
a ``tty'' (terminal) service,
and a service to start new processes (the ``spawn'' service).
Grass is a microkernel, and it runs a complete file service in user space.
(More on this later as well.)

There are three message types:
\begin{enumerate}
\item \texttt{MSG\_REQUEST}: a request message sent from a client process
to a server process;
\item \texttt{MSG\_REPLY}: a reply message sent by a server process,
in response to a request message, to the client process;
\item \texttt{MSG\_EVENT}: ``unsolicited messages'' that can be sent
in an unstructured fashion.  These messages are currently only used to
notify deaths of processes.
\end{enumerate}

Because services are implemented by processes, only seven system calls suffice:

\begin{enumerate}
\item \texttt{gpid\_t sys\_getpid()}: returns the process identifier
of the current process;
\item \texttt{int sys\_recv(type, max\_time...)}: wait for a message of the
given \texttt{type} (a request or an event) and then return it.
\texttt{max\_time} specifies a
time-out in milliseconds, with 0 meaning that the process is willing to
wait for a message indefinitely;
\item \texttt{int sys\_send(dst, type, ...)}: send a message of the
given \texttt{type} (a reply or an event) to process \texttt{dst};
\item \texttt{int sys\_rpc(dst, ...)}: send a request message and wait
for a response;
\item \texttt{void sys\_exit(int status)}: terminates the process with
the given status;
\item \texttt{unsigned long sys\_gettime()}: returns the time since the kernel was
booted in milliseconds;
\item \texttt{void sys\_print(char *string)}: prints the given string, which
is useful when debugging and calls like \texttt{printf()} no longer work.
\end{enumerate}

Typically server processes run a loop in which they wait for a request
message, then perform the requested service, and send a reply.  They
are not required to send a reply before waiting for the next request.
For example, the terminal server can accept requests for writing to
the terminal screen while another client is waiting for terminal input.
A client sends a request message and then immediately blocks
waiting for a response.  It is not possible for a single process to have
multiple outstanding requests.

There are a variety of kernel processes that can roughly be grouped
by the type of interface they implement:

\begin{itemize}

\item Block services: these provide interfaces to read and write disks
by ``inode number'' (partition identifier).
Grass runs two of these: one to support the user level file server, and
one for paging.

\item File services: these provide interfaces to read and write files
by ``file number.''  (A file is identified by the process identifier
of the server and its file number.)  There are currently two file
servers:
\begin{enumerate}
\item the ``tty'' or terminal server that is used to read from the
keyboard or write to the screen;
\item the ``ram file'' server is a very simple file server that keeps
files in memory.  It is used for /tmp as well as to hold some executables
that cannot be paged, such as the main file server that runs in user
space.
\end{enumerate}

\item The ``spawn server'' supports starting new user processes.

\item The ``gate server'' provides an interface to the Earth layer and the
outside world.  For example, through the gate server it is possible to
read files from the underlying Linux or Mac OS X file system, or to read
the absolute time.

\end{itemize}

Other important operating services run in user space:

\begin{itemize}
\item A block server is a storage server that presents a ``block device''
as a simple array of blocks.  A block server can serve multiple block
devices, each identified by an inode number.  A block server can support
the file server listed above but also paging.
Each block server actually consists of a set of ``block layers,'' which
will be explained later.  A block server uses the same interface as the
kernel block servers.

\item The ``block file server'' implements a full-featured file system on top of
a block server.  It uses the same interface as the kernel file servers.

\item The directory server maps
file names to file numbers and provides interfaces to insert, lookup,
or delete mappings.  Directories are kept on files.  While processes
could in theory update directories themselves using the file interface,
doing so would lead to all kinds of race conditions when multiple
processes try to do so.  The directory server serializes all those
updates.

\item The ``sync server'' synchronizes the caches of all files every 30 seconds.

\item The ``init server'' is the first user level process that runs: it
initializes the file system and runs the login process, which in turn
launches a shell.
\end{itemize}

\section{Building and Running EGOS}

First install the code in a directory.  You can build EGOS either for
64-bit Linux or 64-bit Mac OS X.
(We recommend Linux---it has better debugging tools like \texttt{gdb}
readily available.)
Run `\texttt{make}', and everything should build just so.
If you make changes to any source file, you need to run \texttt{make} again.
However, there are bugs in the ``Makefile'' and sometimes some files do
not get recompiled properly.  In that case, run:
\begin{verbatim}
make clean
make
\end{verbatim}

The easiest way to launch EGOS is to run `\texttt{make run}'.
A login prompt should come up, and you can log in using the guest
account ``guest'' with password ``guest''.
A ``shell'' should come up, and you can type in commands such as
\begin{itemize}
\item \texttt{ls}: list the current directory;
\item \texttt{echo hello world}: print ``hello world'';
\item \texttt{cat README.md}: print file ``README.md'';
\item \texttt{shell}: launch another shell, nested in the current one.
(The number in the prompt is the process identifier, so you can tell
which is which.  As with Linux and Mac OS X, if you type
`\texttt{$<$ctrl$>$d}', you indicate an `END-OF-FILE' terminating the shell.)
\item \texttt{shutdown}: synch all files (if using a write-back cache) then kill all processes and quit (may cause kernel crash due to dependencies).
\end{itemize}

If you type `\texttt{$<$ctrl$>$l}', all processes will be listed,
and you should see each of the kernel processes mentioned above  as well as
some user processes.  If you're done with EGOS, type `\texttt{$<$ctrl$>$q}'
and EGOS will quit without synching for write-back cache.
(Note that the \texttt{$<$ctrl$>$} commands are
interpreted by the terminal server, not the shell.)

For example, if you run \texttt{ls -l} you should see something like this:

\begin{verbatim}
S:I    SIZE    NAME
====================
5:20   128    ..dir
5:6    160    ...dir
5:23   2682   README.md
5:24   64     script.bat
\end{verbatim}

Each line lists an entry in the current working directory.  In front of
each file name, ``\texttt{S:I}'' indicates the server (by process
identifier), and the ``file number'' of the file.  Together, these
identify a file, so we call the pair a \emph{file identifier}.
Most names end in ``.$<$\emph{type}$>$'', indicating the file type.
For example, directories have a ``.dir'' suffix.
By convention, ``..dir'' is the current directory and ``...dir'' is the
parent directory.

In this case, process~5 is the file server.  You can see that if you
type `\texttt{$<$ctrl$>$l}', which should print something like this:

\begin{verbatim}
Processes (current = 2):
PID   DESCR    UID STATUS      OWNER ALARM   EXEC
   1: main       0 AWAIT EVENT     1
   2: tty        0 AWAIT REQST     1
   3: blockfil   0 AWAIT REQST     1
   4: ramfile    0 AWAIT REQST     1
   5: diskfile   0 AWAIT REQST     1
   6: dir        0 AWAIT REQST     1
   7: spawn      0 AWAIT REQST     1
   8: user       0 AWAIT EVENT     1         5:7
   9: user       0 AWAIT REQST     8         5:10
  10: user       0 AWAIT EVENT     8         5:8
  11: user     666 AWAIT     2    10         5:19
\end{verbatim}

In this case there are 11 processes, with process~2 being the
current process (the ``tty'' or terminal server).  None of them are actually
running---they're all waiting for something (a message to arrive).
Processes~1, 8, and~10 are waiting for death notifications of the processes
that they own (\texttt{AWAIT EVENT}).
Process~1 is the first kernel process for EGOS.
Process~8, the ``init'' process (executable ``/etc/init.exe''),
is the first and only user process that is spawned during initialization
by the Grass kernel.  ``init'' then spawns the ``login'' process
(``/etc/login.exe'') using the spawn server (process~7), which in turn
launches the shell (``/bin/shell.exe'').
Process 11 is the shell, which is waiting for a response from process~2,
the tty server.
Most other processes are servers waiting for a request (\texttt{AWAIT REQST}).

Processes 8 through 11 are user processes.  The listing contains the
file identifier of the executable they are running.  You can find what
those are by running \texttt{ls /bin} and \texttt{ls /etc}.

\section{Directory Organization}

The directory under Linux or Mac OS X is organized as follows:
\begin{itemize}
\item \texttt{.}: the top level directory contains the Makefile and,
when run, also the executables that are generated.  The Makefile invokes
three other Makefile (in \texttt{src/make}):
Makefile.apps (to generate a library called \texttt{lib/libgrass.a})
as well as a set of applications (in the \texttt{bin} directory)),
Makefile.grass (to generate the Grass kernel called \texttt{build/grass/k.exe}),
and Makefile.earth (to generate the Earth virtual
machine, which is \texttt{build/grass/earthbox}).
Note that EGOS executables end in the \texttt{.exe} suffix.  These will
be loaded into the EGOS file system.
\item \texttt{src/make}: Makefiles for building the EGOS system;
\item \texttt{src/earth}: the source files for the Earth layer;
\item \texttt{src/grass}: the source files for the Grass layer;
\item \texttt{src/include}: C standard include files;
\item \texttt{src/h}: EGOS-specific include files;
\item \texttt{src/lib}: the source files for the library for applications;
\item \texttt{src/block}: the source files for the various block layers;
\item \texttt{src/apps}: the source files for the various apps;
\item \texttt{src/tools}: some useful tools;
\item \texttt{bin}: EGOS executables;
\item \texttt{lib}: EGOS runtime libraries and linkable objects;
\item \texttt{storage}: virtualized storage devices;
\item \texttt{tcc}: Tiny C Compiler;
\item \texttt{usr}: EGOS home directories;
\item \texttt{docs}: source files for documents.
\end{itemize}

All these are copied into the EGOS file system when EGOS is being initialized.

\section{The Earth Layer}

The Earth layer emulates the underlying hardware devices that are not
actually present inside a Linux or Mac OS X process.  The Earth layer
is designed to support different operating system kernels running on
top of it, although we shall focus on the case where Grass is running
on top.  Earth currently only supports a single core.

Earth creates two address spaces.  The kernel is loaded at address
0xA,000,000,000, while user process will run at address
0x9,000,000,000.  The kernel can interface with Earth through
the ``earth'' interface (see \texttt{src/earth/intf.c}).

\subsection{The TLB}

The TLB (Translation Lookaside Buffer) is a fixed-size associative
array that essentially maps virtual addresses to physical addresses.
It also supports protections bits (\texttt{P\_READ}, \texttt{P\_WRITE},
\texttt{P\_EXEC}), although the Grass layer currently simply sets all
bits whenever it maps a page.
When the current process makes an access to a virtual address, it may
or may not be mapped.
If it is mapped, the TLB specifies which physical page is associated
with the virtual page in which the virtual address falls.
If it is not mapped (or the protection bits do not allow access),
an \texttt{INTR\_PAGE\_FAULT} signal is thrown.
The TLB is not backed by a hardware-implemented page table, so it is
up to the operating system to handle the interrupt, add a new entry
to the TLB, and then return from interrupt.
The operating system is free to implement single- or multi-level page
tables, segmented or inverted page tables, or whatever it likes.
(Grass implements a simple single-level level page table.)
The TLB does not know about different processes, so it needs to be
``flushed'' whenever context switching between processes.

\subsection{Interrupts}

There are currently just four types of interrupts:

\begin{itemize}
\item \texttt{INTR\_PAGE\_FAULT}: a page fault, as discussed above.
It specifies the address at which the page fault happens, but
unfortunately nothing else.  (But you can guess based on the protection
bits of the page.)  The kernel page fault handler should either
map the address or decide to kill the process.
\item \texttt{INTR\_CLOCK}: clock interrupts currently happen 100 times
per second.  They are needed for the kernel to get control back from
a running user process.
\item \texttt{INTR\_IO}:
This means some I/O is ready.  Which kind is unfortunately not specified,
however, each device can separately specify a callback function to be invoked
when it has I/O available for processing.  The \texttt{INTR\_IO} interrupt is
intended to allow a scheduling decision to happen.
\item \texttt{INTR\_SYSCALL}:
A system call interrupt is thrown when a user process want to execute
one of the system calls listed above.  System calls are implemented with
an illegal x86 instruction (\texttt{.long 6} to be precise).  The interrupt
handler is passed a pointer to a \texttt{struct syscall} record that
specifies what system call is invoked, and also allows for a return value
to be sent back to the user process.
\end{itemize}

Interrupts are \emph{only} enabled in two cases: when EGOS is executing
in user space, or when EGOS is waiting for I/O.  Thus, when the
operating system (Grass) is executing, interrupts are disabled.
Interrupts are never ``dropped,'' even when disabled.  Once enabled,
they fire and need to be handled.

\subsection{User Space vs. Kernel Space}

Ok, there really is no user space and kernel space---it's all make
believe because a Linux or Mac OS X process runs in user space only.
But the emulation is not bad.
Earth reserves a range of address space, \texttt{VIRT\_BASE} to
\texttt{VIRT\_TOP}, to be used in ``user space''.
There is also a range of ``physical memory'' reserved that can be
used for user process pages.

When not executing in user space, the process state is saved on the
``interrupt stack'' stack of the process.
(In Grass, each user process ends up with three stacks: the kernel
stack, the interrupt stack, and the user stack.  Only the user stack
is paged.)
In Linux and Mac OS X, one cannot change the interrupt stack while
the operating system thinks the process is executing on it.  Therefore
the contents of the interrupt stack must be saved and restored by
copying.
The initial interrupt stack is created in \texttt{main()}
(in ``grass/main.c()'') by explicitly accessing address~1.
(By using \texttt{ctx\_switch()}, we also are able to move back
off the interrupt stack so we can easily copy it.)
This interrupt stack then becomes the initial interrupt stack of
all user processes.

\subsection{Devices}

Earth supports a few simple devices:

\begin{itemize}
\item \texttt{clock}: the clock device that interrupts 100 times per second;
\item \texttt{disk}: a disk device for storing and retrieving blocks;
\item \texttt{gate}: a pseudo-device that is useful for interactions with the ``outside world'';
\item \texttt{log}: allows the kernel to log or print messages;
\item \texttt{tty}: the terminal device for reading from keyboard and writing to the screen.  It interrupts when there is input from the keyboard;
\item \texttt{udp}: a device for sending and receiving UDP packets, which
interrupts when a UDP packet has been received.  (Not currently used by Grass.)
\end{itemize}

\section{The Grass Layer}

The Grass layer runs on top of Earth.  It implements multi-processing,
messages passing, paging, files, and directories.

\subsection{A Grass Kernel Process}

A Grass kernel process has a kernel stack and is ``non-preemptive'':
a process runs until it explitly yields
to another process using function \texttt{proc\_yield()}.
Function \texttt{proc\_yield()} searches for another process to run
and then executes a so-called \emph{context switch} by saving its
registers on its stack followed by restoring the registers that were
previously saved by the other process.  The other process is now running.

A process is always in one of the following states:
\begin{itemize}
\item \texttt{PROC\_RUNNABLE}: ready to run or in fact running.  Because EGOS
is single core, only one process can run at a time.
\item \texttt{PROC\_WAITING}: waiting for a message to arrive.  Currently
a process has to specify the type of message that it is waiting for.
(Separate state variables indicate what type of message the process is waiting for.)
\item \texttt{PROC\_ZOMBIE}: this is a process that is dying but not quite
dead.  When a zombie process context switches to another process, the
latter cleans up the zombie once and for ever.
\end{itemize}

The current process, which may or may not be runnable, is pointed to by
pointer \texttt{proc\_current}.  This pointer should always be valid.
All runnable processes, except any that is current, are
on the \texttt{proc\_runnable} queue.  That is, processes are scheduled
in round-robin order.
When a process yields, it tries to dequeue a runnable process.  If there
is none, the process either returns (if itself is still runnable), or
it waits for I/O interrupts or timeouts
(using Earth function \texttt{intr\_suspend(timeout)}).

\subsection{A Grass User Process}

A Grass user process is simply another kernel process, but one that
has a page table and an interrupt stack attached to it.
Its code is in \texttt{user\_proc} in file ``grass/spawn.c''.
The page table maps ``pages'' to ``frames.''  It is a simple one-level
page table, with one entry for each virtual page.  The page table
starts out empty.  When the process runs, page faults happen whenever
the process accesses an unmapped page in its virtual address space
range.  The job of function \texttt{proc\_pagefault()} is to allocate
a frame, save it in its page table, add a new entry to the TLB, and
then return from the interrupt.  For pages in the code and data section
of the process, the corresponding frames are initialized by reading
the data out of an executable file.  Other frames are zero-initialized.

The top of the user stack of the process is initialized with a block
of data that contains the arguments to the process, as well as the
``Grass environment'' pointed to by macro \texttt{GRASS\_ENV}
(see file \texttt{src/h/egos/syscall.h}), which has the following fields:

\begin{itemize}
\item \texttt{self}: the process identifier of the process, which should
be accessed by function \texttt{sys\_getpid()};
\item \texttt{servers}: an array of process
identifiers for various handy server processes;
\item \texttt{stdin}, \texttt{stdout}, \texttt{stderr}: file identifiers
for standard input, output, and error output;
\item \texttt{cwd}: file identifier of the current working directory;
\item \texttt{argc}, \texttt{argv}, \texttt{envp}: point to the arguments
of the process.
\end{itemize}

A process cannot directly access its own virtual memory when executing
in kernel mode (because page fault interrupts are disabled in kernel
mode).  Instead, it should use function \texttt{copy\_user()}.

\subsection{Grass File Systems}

Grass currently offers two file systems.  The first, ``ramfile,'' 
is a simple file system that keeps each file in contiguously
allocated memory.  It is simple, but not durable, and appending
to a file may require the entire file to be copied to new place.
It runs as a kernel process.

The other file system, called ``bfs'' (for ``block file server'')
provides the same interface
but is based on a sophisticated layered storage subsystem.
It runs in user space, and is layered over a ``block server.''

A block server in turn provides a layered abstraction of a ``disk.''
Each disk is partitioned into one or more ``i-nodes'', and each i-node
consists of a sequence of blocks, numbered $0, 1, ...$. A block is 
typically a power of~2 bytes.
The block interface allows one to read or write a block
at a time---that is, there are no interfaces to access individual bytes.

In Grass, disks are virtual.
Each ``disk module'' implements the same block interface.  The bottom
layer would typically be a kernel block server that actually keeps
blocks in a file.  One can also use a ``ram block server'' if you
simply want to keep the data stored in memory.
Grass provides a wide variety of disk module types, including:

\begin{itemize}
% \item LRU disk module: implements an LRU write-through cache.
\item Check disk module: keeps a hash of all the blocks and checks to make sure that every block read has the same contents as the last time it was written.
\item CLOCK disk module: implements a write-through cache based on the CLOCK algorithm.
\item Combine disk module: combines an array of underlying block stores into a single block store.
\item Debug disk module: prints all the read and write operations that flow through it.
% \item FAT disk module: TODO
% \item File disk module: sits on top of the underlying POSIX file system.
\item Map disk module: forwards calls to a particular inode in the block store below.
\item Partition disk module: cuts up an underlying disk module into multiple fixed size partitions (each is identified by ``inode number'').  You can layer a disk module on top of each of those partitions.
\item Protocol disk module: this module allows access to a remote disk module using Grss RPC.
\item RAID0 disk module: this is layered on top of multiple underlying disk modules and simply stripes operations across them for better throughput.
\item RAID1 disk module: this is layered on top of multiple underlying disk modules and mirrors everything written on all disks for better fault tolerance, and distributes read operations for better throughput.
\item RAM disk module: this a disk module that stores the blocks in memory, which has good performance but poor durability.
\item Statistics disk module: keeps track of the rates of block read and block write operations that flow through it.
\item Tree disk module: like the Partition disk module, but cuts up a disk module into multiple variable size ``partitions,'' each of which is a tree of blocks organized under a single inode.
\end{itemize}

You can take these disk modules and stack them in any way you like,
creating a DAG (Directed Acyclic Graph) of disk modules.  You may
even want to stack multiple instantiations of the same disk module type.

The tree disk module is the one that looks closest to a Linux or
Windows file system.  In fact, it would be good if eventually we had
an EXT4 disk module, a FAT disk module, and so on, but for now there
is only the tree disk module.  It implements each partition as a tree of
blocks.

The tree disk module does not (knowingly) keep track of which inodes are
free or allocated, how many bytes exactly they contain, when it was last
written, or any of that information.  The block file server keeps this
information on partition~0 of the tree disk, and thus implements the file
interface.  But the file server would need no changes if it ran on top
of an EXT4 disk module, say.

\section{Service Protocols}

Looking up file names, reading and writing files, and spawning processes
are all implemented by a specific protocol between a client that
requests a service and a server that implements the service.
This section lists both the service interfaces that are available to
clients and some specifics of the underlying protocols.
The interfaces and the protocols are specified in the ``shared'' directory.

\subsection{File Interface and Protocol}

To create a file, use the following interface:

\begin{verbatim}
bool_t file_create(gpid_t server, gmode_t mode,
                   /* OUT */ unsigned int *p_fileno);
\end{verbatim}

Here \texttt{server} is the process identifier of the file server
and \texttt{mode} contains the access control bits.
It returns \texttt{True} if successful, and if so it returns a file number in \texttt{*p\_fileno}.

To read a file identified by file identifier \texttt{fid}, a client invokes:

\begin{verbatim}
bool_t file_read(fid.server, fid.file_no, offset,
                 /* OUT */    *data,
                 /* IN/OUT */ *psize);
\end{verbatim}

Argument \texttt{offset} specifies the byte offset into the file,
\texttt{data} points to a region of memory in which the data is
to be copied, and \texttt{*psize} indicates the size of that region
and thus the maximum amount of data to be read.
The function returns either \texttt{False} if there is an error,
or \texttt{True}, in which case \texttt{*psize} is filled with the
amount of data read by the function.  If \texttt{*psize == 0}, then
End-Of-File (EOF) has been reached.

To write a file identified by \texttt{fid}, a client invokes:
\begin{verbatim}
bool_t file_write(fid.server, fid.file_no, offset,
             /* IN */ *data, size);
\end{verbatim}

where once again \texttt{offset} is a byte offset into the file,
\texttt{data} points to a region of memory containing the data to
write, and \texttt{size} is the size of that region in bytes.

Other handy interfaces:

\begin{verbatim}
bool_t file_getsize(gpid_t svr, unsigned int file_no,
                               unsigned long *psize);
bool_t file_setsize(gpid_t svr, unsigned int file_no,
                               unsigned long size);
bool_t file_delete(gpid_t svr, unsigned int file_no);
bool_t file_sync(gpid_t svr, unsigned int file_no);
\end{verbatim}

The file service protocol uses request and reply messages
specified in ``$<$egos/file.h$>$'' (which you can find in \texttt{h/egos/file.h}).
The protocol is implemented by
file servers and also the tty server.  The tty server ignores the offset,
and uses file number~0 for standard input, file number~1 for standard output,
and file number~2 for error output.

\subsection{Directory Interface and Protocol}

Directories are simply maintained in ordinary files (by convention
named with the ``.dir'' extension).  However, because multiple processes
might simultaneously try to update the same directory, update operations
to insert and remove entries in a directory all go through the
directory service, whose process identifier is available to a user
process as \texttt{GRASS\_ENV-$>$dir}.  By convention, most file services
will use file number~1 to contain a ``root directory'' for that file service.

An entry in a directory file has the following format
(see ``$<$egos/dir.h$>$''):

\begin{verbatim}
struct dir_entry {
    fid_t fid;
    char name[DIR_NAME_SIZE];
};
\end{verbatim}

Some entries in a directory may be all zero bytes, indicating that
the entry is not currently in use.

The interface to the directory service is as follows:

\begin{verbatim}
bool_t dir_lookup(gpid_t svr, fid_t dir, char *path,
                               /* OUT */ fid_t *pfid);
bool_t dir_insert(gpid_t svr, fid_t dir, char *path,
                               fid_t fid);
bool_t dir_remove(gpid_t svr, fid_t dir, char *path);
\end{verbatim}

Here \texttt{svr} is typically \texttt{GRASS\_ENV-$>$dir},
\texttt{dir} is the file identifier of the directory, and \texttt{path}
is a (relative) path name.  (The current directory server does not yet
support actual path names with slashes in them.)
There is no \texttt{dir\_list} interface---the client is supposed to simply
read the corresponding file using \texttt{file\_read()}.

The same header file specifies the format of the protocol messages.

\subsection{Spawn Interface and Protocol}

To spawn a new process, a client invokes:

\begin{verbatim}
bool_t spawn_exec(gpid_t svr, fid_t executable,
               char *argb, unsigned int size,
               /* OUT */ gpid_t *ppid);
\end{verbatim}

Here \texttt{svr} is typically \texttt{GRASS\_ENV-$>$spawn},
\texttt{executable} the file identifier of the file containing
the executable (with a format specified in ``$<$egos/exec.h$>$''),
and \texttt{argb} is the initial contents of the user stack
consisting of \texttt{size} bytes.

\texttt{argb} should contain the process argument and the
``grass environment.''  It can be generated with the following
function:

\begin{verbatim}
bool_t spawn_load_args(struct grass_env *ge_init,
           int argc, char **argv,
           /* OUT */ char **p_argb, unsigned int *p_size);
\end{verbatim}

Here \texttt{ge\_init} is the initial grass environment, which
is usually just \texttt{GRASS\_ENV}.  The spawn server will overwrite
field \texttt{self} with the process identifier of the new process.
\texttt{argc} and \texttt{argv} specify the argument vector.
The result is allocated by \texttt{malloc()} (and must be freed when
no longer in use) and placed in \texttt{*p\_argb}.  \texttt{*p\_size}
is filled with the size of this region.

There is no \texttt{spawn\_wait()} interface---instead a process can
simply wait for \texttt{MSG\_EVENT} messages, which have type
\texttt{struct msg\_event} and contain the process identifiers and
exit statuses of the processes that die (see ``$<$egos/syscall.h$>$'').
Status \texttt{-1} is used for the case when the process dies
because it accessed an illegal address.

\subsection{Block Interface and Protocol}

The block interface is for disk modules.  Each server
can manage multiple disk module instances that are identified by inode number.
The interface is similar to the file interface but manipulates
blocks of \texttt{BLOCK\_SIZE} bytes:

\begin{verbatim}
bool_t block_read(gpid_t svr, unsigned int ino,
        unsigned int offset, void *addr);
bool_t block_write(gpid_t svr, unsigned int ino,
        unsigned int offset, const void *addr);
bool_t block_getsize(gpid_t svr, unsigned int ino,
        unsigned int *p_nblocks);
bool_t block_setsize(gpid_t svr, unsigned int ino,
        unsigned int nblocks);
bool_t block_sync(gpid_t svr, unsigned int ino);
\end{verbatim}

Note that the \texttt{offset} is measured in blocks, not bytes.

\subsection{Password Service and Protocol}

The password file ``/etc/passwd'' should not be directly updated for
two reasons.  First, concurrent updates could leave the password file
corrupted.  Second, it is of course a security issue if everybody could
update the password file.  The password file is therefore updated by
the ``password server,'' which runs in user space.  The \texttt{passwd}
application allows users to update their password.  This application
sends a request to update the password file to the password server.

The password file has a similar format as what is used in Linux, with
one line for each user.
For example, the line for user \texttt{guest} might look like:

\begin{verbatim}
guest:CDS56G1ck?kWezL13Wg3Btei5Hkfad0M:666:/usr/guest
\end{verbatim}

A line consists of four entries:
\begin{enumerate}
\item user name;
\item hashed password;
\item user identifier;
\item home directory.
\end{enumerate}

The hashed password consist of a four character random salt and a 28 character
hash based on a SHA256 computation over the salt and the user's password.
(By the way, using SHA256 is not a great choice because of the availability
of hardware that can quickly compute SHA256 hashes over a large number of
inputs.)

The first line in the password file is for user \texttt{root}, aka the
\emph{superuser}.  The superuser has certain managerial powers, like being
able to read or write any file and kill any process.

\section{The Application Layer}

The sources of the applications in the ``apps'' directory in the host
operating system, while the executables are installed in the \texttt{/bin}
and \texttt{/etc} directories under Grass.  The sources use library
routines that are stored in the host \texttt{src/lib} directory
directory as specified in the previous section.
Each application is loaded with \texttt{lib/crt0.o}, in which \texttt{\_start()}
is the entry point.  The address of \texttt{\_start()} is in the header
of the executable (see ``$<$egos/exec.h$>$'').  Function \texttt{\_start()}
invokes \texttt{main()}.  Should \texttt{main()} return, then it
automatically invokes \texttt{sys\_exit(status)} using the return value
of \texttt{main()}.

The applications are compiled with a standard C compiler (such as gcc or
clang), and loaded with \texttt{lib/libgrass.a}.
The resulting executable has a ``.int'' extension and
is in either ELF format (Linux) or MACHO format (Mac OS X).  Tools in the
\texttt{src/tools} directory then convert these into ``.exe'' files with headers
as specified in ``$<$egos/exec.h$>$''.
The ``.exe'' files are automatically copied into the EGOS file system.
\texttt{main()} (in \texttt{src/grass/main.c}) starts the ``init.exe''
process as mentioned above,
which in turn spawns an instance of \texttt{bin/shell.exe}.

The library supports \texttt{malloc()} and \texttt{printf()} and a
bunch of other common standard C functions.
There is also a rudimentary Posix compatibility layer
(see \texttt{lib/unistd.c}).

\subsection{The Shell}

The shell is a typical command-line interface to the operating system.
The shell accepts one command per line.  A typical command is:
\begin{verbatim}
cat README.md
\end{verbatim}

This will try to locate the file ``cat.exe'' and then contact the
spawn server to run it.  In this case,
function \texttt{int main(int argv, char **argv)}
in ``cat.exe'' will be invoked with two arguments: ``cat'' and ``README.md''.
The shell will then wait for the process to terminate before printing a
new prompt and waiting for the next line of input.

It is possible to run commands in the background by ending the line with
a ``\&''.  So, for example:
\begin{verbatim}
loop&
\end{verbatim}
will run \texttt{loop.exe} but the shell will \emph{not} wait for its
completion.
(\texttt{loop.exe} in \texttt{/bin} is handy for testing.)
It just runs in a loop
for a number of times that amounts to maybe 5-15 seconds.)
If, at a later time, you want to explicitly wait for
commands that may or may not still be running in the background,
the shell as a built-in command:
\begin{verbatim}
wait [pid]
\end{verbatim}
If you specify a process identifier, then the shell will wait for
that process identifier to finish.  If you don't specify a process
identifier, the shell will wait for all background processes to finish.

Currently, the only other built-in command is
\begin{verbatim}
exit [status]
\end{verbatim}
(status 0 is default).
For example, try:
\begin{verbatim}
shell
exit 3
\end{verbatim}
The (parent) shell should print something like
\texttt{Process X terminated with status 3}.
The shell prints a line like this everything it sees a process terminating
with a non-zero status or if it sees a process terminating it was not
explicitly waiting for.

\subsection{Other Apps}

\begin{itemize}
\item \texttt{cat [file ...]}: print the contents of the given files.
If no files are specified, read standard input (terminated by
\texttt{$<$ctrl$>$d}.  Also, the ``--'' file name is interpreted
as standard input.  Returns 1 if a file does not exists.
% \item \texttt{chmod}: TODO
\item \texttt{cc ...}: Yes, there is a C compiler.  Most apps and the
Grass kernel itself can be recompiled under Grass, a sure sign of
its maturity.
\item \texttt{cp file1 file2}: copy \texttt{file1} to \texttt{file2},
creating \texttt{file2} if it didn't exist before.
\item \texttt{echo [args ...]}: simply print the arguments.
\item \texttt{ed file}: simple file editor like Unix v7 \texttt{ed}.  Simpler even.
% \item \texttt{init}: TODO
\item \texttt{kill pid ...}: kill one or more processes.
% \item \texttt{login}: TODO
\item \texttt{loop [\#]}: loop for the given number of iterations in a
simple tight loop.  \texttt{loop 0} means loop for ever.  If no argument
is given, the default loops for about 5-15 seconds.
\item \texttt{ls}: list the contents of the current directory.
Output is in the form \texttt{S:N name}, where \texttt{S} is the
process identifier of the server, and \texttt{N} is a file number.
% \item \texttt{makedir}: TODO
% \item \texttt{mt}: TODO
\item \texttt{passwd}: to change the password of a user.
\item \texttt{pwd}: print working directory name.
% \item \texttt{pwdsvr}: TODO
% \item \texttt{shell}: TODO
\item \texttt{shutdown}: syncs all files (if write-back cache is enabled) and quit EGOS.
\item \texttt{sync}: syncs all files (if write-back cache is enabled).
\end{itemize}

\section{History and Acknowledgments}

Much of EGOS was written by Robbert van Renesse in the summer of 2018
to replace the OS for the Cornell CS4411 class that he was teaching in the
Fall semester of that year.  However, he borrowed some code from earlier
projects he had done.  In particular, the ``block layer stack'' was largely
drawn from code he had written for a CS4410 homework project in Fall 2015.
Yunhao Zhang helped Robbert significantly in developing the educational
aspects of EGOS and in particular refined many of the projects.  He also
wrote the fatdisk block layer, and taught CS4411 by himself in
Spring 2019 and Fall 2020.
Edward Tremel taught CS4411 in Spring 2020 and made various improvements.

Jason Liu (CS'19) took CS4411 in Fall 2018 and contributed cipherdisk.c, a
block layer that encrypts/decryps blocks.  Alice Chen (CS'19) also took the
course that semester and, during Spring 2019, in collaboration with Robbert
and Yunhao, made two important improvements to the block layer.  First, she
changed the block layer API so each layer instantiation can serve multiple
partitions or inodes.  Second, she added support for writeback caching.
She also developed several unit tests for the block layer.
Yizhou Yu helped with porting the Tiny C Compiler to EGOS.
Kenneth Fang and Mena Wang contributed a Unix FS disk layer.

\end{document}
